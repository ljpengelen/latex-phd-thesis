\section{Simple Language of Communicating Objects}\label{sec:SLCO}
We designed a DSL called Simple Language of Communicating Objects (SLCO).
This language provides constructs to specify systems consisting of objects that operate in parallel and communicate over channels.
An SLCO model consists of a number of classes, instances of these classes, and channels.
Instances of classes are called objects.
These objects can be connected to each other by channels. Figure~\ref{fig:slco_ports_example} shows two objects, $\it{a1}$ and $\it{a2}$, that are instances of the same class, $A$, and that can communicate over a channel, $c$.

\begin{figure}[hbt]
 \centering
 \includegraphics[width=.8\textwidth]{figs/slco_ports}
 \caption{Two objects connected by a channel in SLCO}
 \label{fig:slco_ports_example}
\end{figure}

A class describes the structure and behavior of its instances.
A class has ports and variables that define the structure of its instances, and state machines that describe their behavior.
Figure~\ref{fig:slco_ports_example} shows that both instances of the class $A$ have a port $p$.
A state machine consists of variables, states, and transitions.
A transition has a source and a target state, and possibly a guard, a trigger, or an effect.
A guard is a boolean expression that must hold to make the transition from source state to target state.
There are two types of triggers: a deadline and the reception of a signal.
If the amount of time specified by a deadline has passed or a signal is received, the transition that has such a trigger is enabled.
The effect of a trigger is a number of statements that are executed when the transition is taken.
There are statements for assigning values to variables and for sending signals.
Figure~\ref{fig:slco_sm_example} shows an example of a state machine.

\begin{figure}[hbt]
 \centering
 \includegraphics[width=.8\textwidth]{figs/Single_SSingle}
 \caption{State machine in SLCO}
 \label{fig:slco_sm_example}
\end{figure}

%%SLCO started as a simplified version of a part of the UML.
%%The syntax of state machines in SLCO is similar to the graphical syntax of UML's state machine diagrams.
%%A difference between UML and SLCO in this respect is the fact that we explicitly defined a textual syntax for the guards and effects.
