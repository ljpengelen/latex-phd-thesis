\section{Related Work}
\label{sec:iterative-dsl-evolution:Related_Work}
%%%%%%%%%%%%%%%%%%%%%%%%%%%%%%%%%%%%%%%%%%%%%%%%%%%%%%%%%%%%%%%%%%%
%% Het stukje over Deursen2000 gaat niet over evolutie,          %%
%% daarom is het niet nodig dit op te nemen in gerelateerd werk. %%
%%%%%%%%%%%%%%%%%%%%%%%%%%%%%%%%%%%%%%%%%%%%%%%%%%%%%%%%%%%%%%%%%%%
%In~\cite{Deursen2000}, a survey is given of key publications in the area of domain-specific languages.
%The authors also provide some terminology and list some advantages and disadvantages of DSLs.
%They also give some basic DSL design and implementation guidelines.

Freeman and Pryce describe their experiences with the evolution of a DSL~\cite{Freeman2006}.
Their language has evolved from a library on top of Java to an embedded language.
We observed that our DSL has evolved because new target platforms were added.
Their language is an embedded language that does not need target platforms.
Because the language is embedded, it is highly influenced by the host language, Java in this case.
One of the reasons for the evolution of their language is to improve user-friendliness.
This is related to our observation that language evolution is influenced by  model quality.

The evolution of a visual programming language for writing real-time control programs for distributed environments and supporting tools is described by Karaila~\cite{Karaila2009}.
The main goal of this language is to make programs understandable.
The author shows that Lehman's eight laws~\cite{Lehman1997} for characterizing the way software systems tend to evolve are not restricted to software systems, but apply to language evolution as well.
The main difference between this language and SLCO is the scale.
The language studied by the author has been under development since 1988 and the last major development step dates from 2003.
Also the approach of the study is different.
Karaila tries to fit his experiences within a framework, whereas we try to distill general lessons from our own experience.

An analysis of over twenty cases where domain-specific languages were designed to implement a model-driven engineering approach is presented by Luoma et al.~\cite{Luoma2004}.
The data for the analysis was gathered by means of interviews and discussions with people involved in constructing the DSLs.
They identified and categorized four influences on defining DSLs: domain concepts, generated output, look and feel of the system built, and expression of variability.
One of the differences with our work is, again, the scope of the study.
They studied the definition of over twenty languages, of which some have been used for several years already.
We defined only one language, but our focus is on the evolution of the language.
Another important difference is that in their case the target platform for the applications developed with the DSL was already chosen, whereas we added multiple target platforms on the fly.

%%%%%%%%%%%%%%%%%%%%%%%%%%%%%%%%%%%%%%%%%%%%%%%%%%%%%%%%%%%%%%%%%%%%%%%%%%%%%%%%%%%%%%%%%%%
%% Is het stukje over Arango1989 wel zo'n sterke observatie?                             %%
%%                                                                                       %%
%% Wellicht is het zo dat het initieel ontwerp van een DSL sterk be�nvloed wordt         %%
%% door het domein en dat de evolutie het juist verder daar vanaf brengt.                %%
%% Dat zou dan dus veroorzaakt kunnen worden door het relatief stabiele probleem domein. %%
%%%%%%%%%%%%%%%%%%%%%%%%%%%%%%%%%%%%%%%%%%%%%%%%%%%%%%%%%%%%%%%%%%%%%%%%%%%%%%%%%%%%%%%%%%%
%In~\cite{Arango1989}, a conceptual framework for domain analysis is described.
%One of the author's observations is that problem domains evolve gradually.
%This is in line with our observation that the evolution of a domain-specific language is influenced by (the evolution of) the problem domain.

A domain-specific visual language that aims at expressing the evolution of domain-specific visual languages is developed by Sprinkle and Karsai~\cite{Sprinkle2004}.
This language and accompanying tools assist in the co-evolution of domain, DSL and models created using the DSL.
They claim that what sets the evolution of domain-specific languages apart is that its primary aim is not backward compatibility but domain specificity.
This is in line with our observation that the evolution of a domain-specific language is influenced by the evolution of the problem domain.
They focus on tools for supporting DSL evolution, rather than researching the causes for language evolution as we do.

Another framework aimed at supporting DSL evolution is described by De Geest~\cite{Geest2008}.
The primary influence on DSL evolution mentioned by the author is use of the DSL.
He claims that by using a DSL, more domain knowledge is gathered, which requires adaptation of the DSL.
Also, new features may be requested to ease the modeling process.
This is in line with our observation that evolution of a language is influenced by model~quality.

Van Deursen and Klint discuss experiences from industrial practice regarding the development of domain-specific languages~\cite{Deursen1998}.
The focus of the article is on the development of a DSL for describing financial products.
They provide guidelines and considerations that should be taken into account when developing a DSL.
Issues as maintainability factors and risks involved in the use of DSLs are also addressed.
The DSL they developed has been used for a few years and during those years it has been concluded that the language needed some changes to increase user friendliness.
This is in line with our observation that evolution of a language is influenced by  model quality.

Karsai et al.\ provide 26 design guidelines for domain-specific languages~\cite{KarsaiDSLGuidelines2009}.
During the development of \SLCO, we noticed that the guideline that advocates the reuse of existing language definitions can be in conflict with the guideline that advocates simplicity.
Reusing a part of the graphical syntax of the \UML and adopting the corresponding distinction between guards, triggers, and statements hampered the understandability of \SLCO and complicated the definition of its semantics.
Additionally, they advise against conceptual redundancy, whereas we introduced multiple versions of conditional signal reception to simplify the definition of the semantics and some of the model transformations.

Paige et al.\ provide 9 principles for the design of modeling languages~\cite{PaigePrinciples2000}.
The principle of uniqueness is related to the concept of conceptual redundancy mentioned above and also advises to avoid duplication of features.
This suggests that conditional signal reception should be revisited during the next redesign of \SLCO.
Our experience concurs with the provided principles, although seamlessness and reversibility are not applicable for our language.
Seamlessness is not applicable because of the variety of target platforms, and reversibility is not applicable because modification of generated code is in conflict with our approach to MDSE.
We aim at generating implementations by refining models using model transformations that are provably correct according to some correctness criterion.
Modifying the resulting code is not supported, because it could hamper its correctness.