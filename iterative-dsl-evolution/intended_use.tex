\section{Mapping to Different Formalisms}
\label{sec:Intended_Use}
We want to simulate, verify and execute SLCO models.
Therefore, we have implemented a number of model transformations that transform an SLCO model into models on different platforms to enable simulation, verification, and execution.
Each of these target platforms has different characteristics.
Table~\ref{tab:platform_characteristics} shows the characteristics of the different platforms.
\begin{table}[hbt]
\centering
\begin{tabular}{l||c|c|c}
              & Synchronous/asynchronous channels & Lossless/lossy channels & Concurrent objects \\
\hline\hline
 SLCO         &   synchronous and asynchronous    &   lossless and lossy    &     unlimited      \\
\hline
 Simulation   &            synchronous            &        lossless         &     unlimited      \\
\hline
 Verification &   synchronous and asynchronous    &        lossless         &     unlimited      \\
\hline
 Execution    &           asynchronous            &          lossy          &      limited       \\
\end{tabular}
\caption{Platform characteristics}
\label{tab:platform_characteristics}
\end{table}
The first column lists the different platforms.
The second column indicates whether communication over channels is synchronous or asynchronous on the corresponding platform.
In case communication is synchronous, both sender and receiver need to be available for message exchange to occur.
In this way, sender and receiver synchronize on message exchange.
In case communication is asynchronous, a sender can send a message even though the receiver is not yet ready to receive it.
The third column indicates whether channels are lossless or lossy.
In case a channel is lossless, a message that is sent will always arrive at the receiver.
In case a channel is lossy, a message that is sent may get lost.
We assume that not all messages get lost.
We also assume that messages, if they arrive, arrive as they are sent, i.e., undamaged.
The fourth column lists the amount of controllers that are available on the platform.
A controller is a device on which an (implementation of) an SLCO object can be deployed.
For SLCO (specification), simulation, and verification this number is unlimited since these controllers are virtual.
On the implementation platform, actual controllers need to be available.
Therefore, this number is limited.
We have also implemented a number of model transformations that transform SLCO models to bridge the semantic platform gaps.
An overview of the model transformations is given in Figure~\ref{fig:trafo_overview}.
\begin{figure}[hbt]
 \centering
 \includegraphics[width=.8\textwidth]{figs/trafo_overview}
 \caption{Overview of the transformations}
 \label{fig:trafo_overview}
\end{figure}
In the remainder of this section, we explain how an SLCO model can be transformed into models for simulation, verification, and execution.
We also explain in detail how we dealt with the differences in platform characteristics.

\subsection{Simulation}
The Parallel Object-Oriented Specification Language (POOSL) is a formal modeling language for simulation and performance analysis~\cite{Theelen2007}.
The syntax of POOSL is based on the syntax of the formal language CCS~\cite{Milner1989}.
The main constructs of POOSL are cluster classes, process classes and data classes.
Each of these classes can be instantiated and the resulting objects can be combined to specify a system.
A cluster class consists of a number of process classes that are connected by channels.
A process class consists of variables, ports, process methods and instantiation parameters.
The ports of a process class can be connected to the ports of other process classes by means of channels.
A pair of process classes connected by a channel can communicate by sending synchronous signals over the channel.
The variables are instances of data classes and are used to store data.
Instantiation parameters are special kinds of variables, whose initial value can be specified.
The behavior of process classes is specified using process methods.
A process method takes a number of parameters as input and can pass output through output parameters.
The behavior of a process method is defined using a number of statements, each of which can be guarded by an expression.
Among others, there are statements for sending and receiving signals, assigning values to variables or parameters, and calling methods of data objects.
These statements can be composed both sequentially and in parallel.
POOSL also has a statement for conditional iteration, a statement for conditional execution, and a statement for non-deterministic execution of a number of statements.

The language is supported by two tools: SHESim and Rotalumis.
SHESim offers a graphical user interface for defining the structure of models as well as editors for editing of the textual definitions of process classes and data classes.
Models can be interactively simulated using the built-in POOSL interpreter of SHESim.
Rotalumis is a command-line tool that can simulate POOSL models at high speed, by compiling them to byte code that is executed on a virtual machine implemented in C++.

Transformation~$T_s$ depicted in Figure~\ref{fig:trafo_overview} transforms SLCO models containing synchronous, lossless channels only to POOSL models.
This transformation transforms each class to a process class.
The state machines of each class are transformed to a number of process methods.
Each process method represents a state.
It contains a number of guarded statements that are executed non-deterministically.
The guards of the statements represent the guards and triggers of outgoing transitions, and the statements represent the effects of those transitions.
Each guarded group of statements ends with a process method call that calls the method representing the target state of the transition represented by this guarded group of statements.

\begin{figure}[hbt]
\centering
\subfloat[Partial state machine]
         {
          \label{subfig:slco_poosl_slco}
          \includegraphics[scale=.2]{figs/slco_poosl_slco}
         }
\hspace{.2\textwidth}
\subfloat[Corresponding \textsc{Promela} code]
         {
          \label{subfig:slco_poosl_poosl}
          \includegraphics[scale=.3]{figs/slco_poosl_poosl}
         }
\caption{Transforming state machines into \textsc{POOSL}}
\label{fig:slco_poosl}
\end{figure}

To transform SLCO models containing asynchronous channels into POOSL, the SLCO model should be transformed such that it contains synchronous channels only.
This is done by transformation~$T_1^\prime$ depicted in Figure~\ref{fig:trafo_overview}.
In this transformation buffers are added, such that asynchronous communication is simulated using synchronous channels.
To transform SLCO models containing lossy channels into POOSL, the SLCO model should be transformed such that it contains lossless channels.
This is done by transformation~$T_2^\prime$ depicted in Figure~\ref{fig:trafo_overview}.
In this transformation, an object is added that represents a lossy channel.
Messages are no longer sent over a channel directly, but via this object.
Upon reception of a message, the object either sends the message to its destination, or it disposes of it.
In this way, a lossy channel is simulated.
An SLCO model with a limited number of controllers can be transformed into POOSL without any problems, since it allows an unlimited number of (virtual) controllers.
Therefore, in transformation~$T_3^\prime$ nothing is changed.

\subsection{Verification}
\label{subsec:verification}
Model checking is an automated verification technique that checks whether a formally specified property holds for a model of a system~\cite{Clarke1999}.
An exhaustive state space search is performed by an automated model checker to determine whether a property holds in a finite state model of a system.
Model checking is commonly used for verifying concurrent systems.
Before automated model checking can be applied, two tasks need to be performed.
First, the system to be verified needs to be specified in a formalism that is accepted by an automated model checker.
Second, the properties that the system should satisfy have to be formulated in a logical formalism that is accepted by an automated model checker.
Usually a form of temporal logic is used for this.
Hereafter, an automated model checker can be used to verify whether the property holds in the model.
If this is not the case, a counterexample in the form of a trace violating the property is often provided.

In order to enable verification of SLCO models, we implemented a model transformation that transforms SLCO models into \textsc{Promela}.
This is transformation~$T_v$ in Figure~\ref{fig:trafo_overview}.
\textsc{Promela} is the input formalism for the model checker \textsc{Spin}~\cite{Holzmann2003}.
\textsc{Spin} can check a model for deadlocks, unreachable code, or determine whether it satisfies an LTL property.

The \textsc{Promela} language has constructs for modeling selections and loops.
These are based on Dijkstra's guarded commands.
\textsc{Promela} has primitives for message passing between processes over channels either using queues, or handshaking.
This enables the modeling of both asynchronous and synchronous communication respectively.
The syntax of expressions and assignments in the \textsc{Promela} language is inherited from \textsf{C}.

Channels in \textsc{Promela} are lossless and can be either synchronous (handshake) or asynchronous (queue).
Therefore, our transformation can transform SLCO models that do not contain lossy channels into \textsc{Promela} code.
Every state machine contained in a class instantiated as an object in an SLCO model is transformed into a \textsc{Promela} process.
State machines can be implemented using an imperative programming style in multiple ways.
We chose to implement them as jump tables using goto-statements.
An example of part of the transformation is shown in Figure~\ref{fig:sm2pomela}.
State~$S_0$ depicted in Figure~\ref{subfig:sm} is transformed into the code depicted in Figure~\ref{subfig:code}.
\begin{figure}[hbt]
\centering
\subfloat[Partial state machine]
         {
          \label{subfig:sm}
          \includegraphics[scale=.2]{figs/sm2promela_1}
         }
\hspace{.2\textwidth}
\subfloat[Corresponding \textsc{Promela} code]
         {
          \label{subfig:code}
          \includegraphics[scale=.3]{figs/sm2promela_2}
         }
\caption{Transforming state machines into \textsc{Promela}}
\label{fig:sm2pomela}
\end{figure}
A state is transformed into an if-statement with a label and a goto-statement that jumps to that label.
Every outgoing transition of state~$S_0$ is transformed into an alternative of the if-statement.
The guard and statements of a transition are transformed into \textsc{Promela} code in a straightforward way.
When the guard condition holds and the statements are executed, the transition to state~$S_1$ has been completed and the state machine is \emph{in} state~$S_1$.
Therefore, the goto-statement that jumps to the label representing state~$S_1$ is added after the transformed statements in the code.

To verify an SLCO model that contains lossy channels, transformation~$T_2^\prime$ should be performed such that messages may be disposed of.
Note that verification of models with lossy channels requires extra care since traces in which all messages are disposed of should be avoided.

\subsection{Execution}
To execute the SLCO models, an implementation platform is required.
We chose to use the Lego Mindstorms platform for this~\cite{Lego}.
The key part of this platform is a programmable Lego brick, called RCX.
This RCX has an infrared port for communication and is connected to sensors and motors for interacting with its environment.
The language used to program an RCX is Not Quite C (NQC)~\cite{Baum2003}.
To enable execution, we implemented a model transformation that transforms SLCO models into NQC code.

NQC, as the name suggests, resembles \textsf{C}.
NQC is severely restricted to fit the Lego Mindstorms platform.
However, there is also an API consisting of a set of constants, functions, values, and macros that provide access to various capabilities of the Lego Mindstorms platform such as sensors, outputs, timers, and communication.

Every state machine contained in a class instantiated as an object in an SLCO model is transformed into an NQC task.
The transformation from state machines to NQC code strongly resembles the transformation from state machines to \textsc{Promela} described in Section~\ref{subsec:verification}.
The only difference is the syntax of the if-statement.

An RCX is a controller as listed in Table~\ref{tab:platform_characteristics}.
RCX's can communicate with each other by sending and receiving messages via an infrared port.
Communication via the infrared port is asynchronous.
Since an infrared signal can be interrupted, this communication is lossy.
This restricts the type of SLCO models that can be transformed.
In order to enable execution of all SCLO models, a number of model transformations, i.e., $T_1$, $T_2$, and~$T_3$ depicted in Figure~\ref{fig:trafo_overview}, were implemented to overcome the platform differences.
In model transformation~$T_1$, the synchronous signals are replaced by asynchronous signals.
In order to maintain semantic equivalence, acknowledgement messages are added for synchronization.
In model transformation~$T_2$, an implementation of the alternating bit protocol~\cite{Bartlett1969} is added to every class.
The state machines are adapted in such a way that they do not send a message directly, but instead offer it to the alternating bit protocol for transmission.
The protocol will ensure that the message will eventually arrive at it destination, under the premise that not all messages get lost.
In this way, lossless message exchange over a lossy channel is achieved.
In model transformation~$T_3$, objects are merged to match the number of available controllers.
Communication between merged objects is implemented using shared variables, since SLCO and NQC do not provide channels internal to objects. 