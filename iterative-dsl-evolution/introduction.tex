\section{Introduction}
\label{sec:iterative-dsl-evolution:Introduction}

The Simple Language of Communicating Objects (\SLCO) gradually evolved from a slightly modified subset of the \UML suited for performance analysis to a domain-specific modeling language (DSML) that offers simulation, verification, and execution of models.
In this chapter, we describe our experiences with the process of developing this DSML and the corresponding model transformations.
By studying the development of a small and easily changeable language and documenting its evolution, we started to investigate the feasibility of a development process for DSMLs in which changes to one artifact have less influence on other artifacts.

The literature provides guidelines for developing domain-specific languages (DSLs)~\cite{Mernik2005,Deursen1998,Deursen2000} and tools that support DSL evolution~\cite{Geest2008,Sprinkle2004}.
There have been numerous reports on evolution of DSLs.
For example, the changes to the SDF language, a DSL used for syntax definition, are described by Visser~\cite{Vis97.thesis}.
Van Beek et al.\ describe the evolution of Chi, a language for modeling and simulating hybrid systems~\cite{Beek2008}.
The evolution of a language used for interchanging models of hybrid systems is described by Van Beek et al.~\cite{Beek2009}.
In most of these publications, however, only the changes themselves are described, and the reasons for these changes are only hinted upon.
On the underlying reasons for DSL evolution literature is scarce and conclusions are scattered.
Additionally, most analysis of DSL evolution provide an a posteriori report only.
In this chapter, we discuss our experiences with an evolving DSML during its iterative design and address research question~\RQ{5}.

\RQFive

As described in Chapter~\ref{chap:SLCO}, we designed a DSML for modeling systems consisting of concurrent, communicating objects.
The structure of a system is modeled using classes, and their behavior is modeled by state machines.
Simultaneously to the development of the DSML, we implemented a number of model transformation to different formalisms: one for simulation, one for execution, and one for verification.
These model transformations were developed consecutively.
Each time a transformation to a new target platform was added, the functionality offered by the existing transformations remained intact.

Furthermore, the way in which the semantics of \SLCO were defined changed over time, which also had its influence on the rest of the definition of the language and its transformations.
At first, the semantics of \SLCO were defined by means of a transformation to the Parallel Object-Oriented Specification Language (\POOSL)~\cite{Theelen2007}.
This transformational description was then replaced with another transformational description based on labeled transition systems.
Later, a formal definition of the operational semantics of \SLCO was given, to enable formal reasoning about the correctness of the model transformations.

We describe the development process and indicate how our DSML has evolved during this process.
We focus on language evolution only and not on the co-evolution of models specified in the DSML such as described in~\cite{Cicchetti2008}.
Even though our DSML is small in terms of the provided number of modeling constructs, we expect that the lessons learned are applicable to projects involving larger DSMLs as well.
In our discussion of related work in Section~\ref{sec:iterative-dsl-evolution:Related_Work}, we show that the conclusions drawn by others overlap with our own.

The remainder of this chapter is structured as follows.
In Section~\ref{sec:iterative-dsl-evolution:Development_Process}, we describe the development process of our DSML and the accompanying model transformations.
Section~\ref{sec:iterative-dsl-evolution:SLCO_Evolution} describes the evolution the language has undergone.
Related work is discussed in Section~\ref{sec:iterative-dsl-evolution:Related_Work}.
In Section~\ref{sec:iterative-dsl-evolution:Conclusions}, we draw conclusions and give directions for further research. 