\section{Discussion}
\label{sec:exploring-boundaries:Discussion}
In Section~\ref{sec:exploring-boundaries:Experiments}, we used the model checker \Spin to illustrate the effect of both coarse-grained and fine-grained sequences of transformations on state spaces.
However, our approach is not limited to one particular model checker.
The refining transformations we implemented take \SLCO models as input and produce \SLCO models as output.
Support for another model checker or a similar tool can be added by implementing a single transformation from \SLCO to the formalism supported by that tool.
%In fact, we implemented such a transformation to a formalism for performance analysis and simulation~\cite{SLCOiterative2010}.

To clearly show the influence of our refining transformations, we used no additional reduction or abstraction techniques.
However, our approach can be combined with such techniques in practical situations.
Using one of the standard state vector compression modes offered by \Spin~\cite{Holzmann1997}, for instance, it is possible to explore larger state spaces.
Using this compression method and the configuration described in Section~\ref{sec:exploring-boundaries:Experiments}, the state space of the timed version of the model of the three conveyor belts can be explored using approximately $15\cdot10^3$ megabytes, instead of $31\cdot10^3$ megabytes.

%%In the worst case, the complete state space has to be explored to verify the validity of a property.
%%Therefore, the results of the experiments discussed in Section~\ref{sec:exploring-boundaries:Experiments}...
Typically, model checking is used to verify whether a property holds for a model of a system.
Because the refining transformations modify the model, properties under investigation may have to change as well.
After adding communication via the ABP to a model, for example, there are unfair traces in the state space representing the behavior that all signals are discarded by the lossy channel.
To consider only the fair traces, a fairness constraint has to be added to the property. 