\section{Conclusions and Future Work}
\label{sec:exploring-boundaries:Conclusions_and_Future_Work}
In this chapter, we proposed an approach using model checking to increase the reliability of code generated from models specified in a DSML called \SLCO.
A model transformation from \SLCO to a language suitable for model checking has been defined to enable model checking of domain-specific models.
Using this model transformation, model checking can be applied on the domain-specific models in every stage of the refinement process.
This chapter addresses research question~\RQ{2} and investigates how the size and complexity of model transformations affects the verifiability of intermediate models produced by sequences of model transformations.
Our experiments show that using fine-grained sequences of transformations enables automatically generating more concrete models that are still suitable for explicit state-space exploration in comparison to coarse-grained sequences of transformations.
Furthermore, even more concrete models can be obtained by applying transformations to part of a model only.

We conducted experiments to validate our approach on multiple cases related to \SLCO.
The results show that it is possible to validate models that are more concrete when fine-grained sequences of transformations are applied.
In other words, reducing the size and complexity of the refining model transformations improved the verifiability of the intermediate models produced by such sequences.
Additionally, since the transformations used to compose fine-grained sequences tend to be smaller than those of coarse-grained sequences, it is easier to locate defects in them.
Another advantage of these transformations is their increased reusability in comparison to the transformations used to form coarse-grained sequences.

As discussed in Section~\ref{sec:exploring-boundaries:Discussion}, reduction techniques such as partial order reduction and state vector compression can be applied to a verification model.
Additionally, reduction may be applied to domain-specific models.
Models in our DSML consist of state machines, and therefore, algorithms for state machine composition~\cite{Holzmann1991} may be applicable.
Additional research is needed to assess whether reducing the number of state machines in a model leads to smaller state-spaces.

We consider applying the approach to larger models and more complex sequences of transformation to be an interesting direction for future work.
The cases on which we applied our technique are rather small, and so are the sequences of transformations.
However, we believe that these small examples already show the advantages of the proposed approach.
Although the systems under investigation might be larger and more complex in practise, their size and complexity does not bound the verifiability of intermediate models.
Instead, the verifiability of these models is bounded by the practical limitations of the hardware used to perform state-space exploration, which determines when state-space explosion causes problems.
Thus, our approach of generating the most concrete models whose state space is still suited for state-space exploration is also applicable on larger and more complex problems.

Model checking is one way of increasing the reliability of systems created in an MDSE process.
Another way to do this is using formal correctness proofs.
When correctness of model transformations can be formally proven, model checking is no longer required to validate intermediate results.
It would then suffice to validate the initial model only.
Formally proving model transformations requires that the semantics of source and target language are formally defined.
Since a lot of DSMLs have an informal semantics only, the correctness of model transformations related to such DSMLs cannot be proven.
Therefore, model checking intermediate models may still be required. 