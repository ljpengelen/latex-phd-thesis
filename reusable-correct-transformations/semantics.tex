\subsection{Operational Semantics of \SLCO}
\label{sec:reusable-correct-transformations:operational_semantics}
For an arbitrary model $m$, the SOS rules that define the semantics of \SLCO generate the complete behavior of the model in the form of an LTS.
An LTS is a tuple $(S, \Lambda, \rightarrow, i)$, where $S$ is a set of configuration, $\Lambda$  is a set of labels, $ \rightarrow \subseteq S \times \Lambda \times S$ is a ternary relation of labeled transitions, and $i\in S$ is the initial configuration.
In our case, for an \SLCO model~$m=\it{mn}~\it{obj^*}~\it{class^*}~\it{chan^*}$,
with \VariableNames the set of all variable names occurring in $m$,
\StateMachineNames the set of all state machine names of $m$,
\StateNames the set of all state names of these state machines,
and~\ConstantExpressions the set of all constant expressions,
the configurations of the LTS generated for $m$ are tuples $\langle m, \ssms, \vglob, \vloc, \buf \rangle$, where
%
$\ssms : \ObjectNames \pfun (\StateMachineNames \pfun \StateNames)$ is a function that indicates current states of the state machines of the objects in $m$,
%
$\vglob : \ObjectNames \pfun (\VariableNames \pfun \ConstantExpressions)$ is an evaluation function that assigns values to the (global) variables of the objects in $m$,
%
$\vloc : \ObjectNames \pfun (\StateMachineNames \pfun (\VariableNames \pfun \ConstantExpressions))$ is an evaluation function that assigns concrete values to the (local) variables of the state machines of the objects of model $m$,
%
and $\buf : (\ChannelNames \times \ObjectNames \times \ObjectNames) \pfun (\SignalNames \times \it{SEQ}(\ConstantExpressions)) \cup \{\nil\}$ represents the content of the one-place buffers corresponding to the asynchronous channels in $m$.
The content of a buffer can be  \nil, denoting an empty buffer, or a tuple consisting of a signal name and a sequence of constant expressions.
Two buffers are associated to each bidirectional, asynchronous channel, one for each direction.

For the LTS of $m$, $\it{LTS}(m)$ in short, the initial configuration conforms to the following constraints.
First, all the buffers assigned to asynchronous channels are initialized to $\textbf{nil}$.
Second, all the state machines are in their initial state.
Third, all variables are initialized to values respecting their types.

The transitions of $\it{LTS}(m)$ are obtained as the least relation deduced from the SOS rules.
A transition can be labeled $\epsilon$, $\it{vn\assignop ce}$, $\it{sgn}(\it{ce^*})$, $\textbf{send}~\it{sgn}(\it{ce^*})$, $\textbf{receive}~\it{sgn}(\it{ce^*})$, or $\textbf{lost}~\it{sgn}(\it{ce^*})$, where $\it{vn}$ denotes a variable name, $\it{ce}$ a constant expression, $\it{sgn} \in \SignalNames$ a signal name, and $\it{ce^*}$ a sequence of constant expressions.
Transitions labeled $\epsilon$ correspond to transitions in \SLCO models without a statement or transitions that have a blocking statement represented by a Boolean expression.
Transitions labeled $\it{vn\assignop ce}$ represent assignments.
The label $\it{sgn}(\it{ce^*})$ represents synchronous communication between two objects, whereas the labels $\textbf{send}~\it{sgn}(\it{ce^*})$, $\textbf{receive}~\it{sgn}(\it{ce^*})$, and $\textbf{lost}~\it{sgn}(\it{ce^*})$ denote sending, receiving, and losing signals during asynchronous communication.

The overall behavior of a model is defined by the behavior of its composite elements, the objects and the channels.
The statements that the objects can execute and the way they interact via the channels determine the dynamics of the model.
The contributed activities of an object are deduced from the specification of its class, and the activities of a class are deduced from the specifications of its state machines.
At the most elementary level of the structure, the behavior of each state machine is derived from its transitions.

%%%%%%%%%%%%%%%%%%%%%%%%%%
%%      Transitions
%%%%%%%%%%%%%%%%%%%%%%%%%%
\begin{figure}[hbt]
\centering
{\fontsize{7pt}{0pt}
$
\boxed{
\begin{array}{c}
\sosrule{
\langle \it{e^*}, \vvars, \vvars' \rangle
\ExpressionsRelation{}
\it{ce^*}
}{
\langle \it{tn}~\textbf{from}~\it{sn}~\textbf{to}~\it{sn'}~\textbf{send}~\it{sign}(\it{e^*})~\textbf{to}~\it{pn}, \it{sn}, \vvars, \vvars' \rangle \\
\TransitionsRelation{\textbf{send}~\it{sgn}(\it{ce^*})~\textbf{to}~\it{pn}}
\langle \it{sn'}, \vvars, \vvars' \rangle
} \\
\\
\sosrule{
\langle \it{ce^*}, \it{vn^*}, \vvars, \vvars' \rangle
\MultipleSubstitutionRelation{}
\langle \vvars'', \vvars''' \rangle, \quad
\langle \it{e}, \vvars'', \vvars''' \rangle
\ExpressionRelation{}
\textbf{true}
}{
\langle \it{tn}~\textbf{from}~\it{sn}~\textbf{to}~\it{sn'}~\textbf{receive}~\it{sgn}(\it{vn^*}~|~\it{e})~\textbf{from}~\it{pn}, \it{sn}, \vvars, \vvars' \rangle \\
\TransitionsRelation{\textbf{receive}~\it{sgn}(\it{ce^*})~\textbf{from}~\it{pn}}
\langle \it{sn'}, \vvars'', \vvars''' \rangle
}
\end{array}
}
$
}
\caption{A subset of all deduction rules for transitions}
\label{fig:reusable-correct-transformations:transitions}
\end{figure}

Figure~\ref{fig:reusable-correct-transformations:transitions} shows some of the SOS rules that relate a single transition in an \SLCO model to a transition on the LTS level.
If an expression~$e$ evaluates to constant expression~$ce$ with respect to evaluation functions~$v_1$ and~$v_2$, we write $\langle e, v_1, v_2 \rangle \ExpressionRelation{} ce$.
Similarly, we write $\langle e^*, v_1, v_2 \rangle \ExpressionsRelation{} ce^*$ for a sequence of expressions~$e^*$.
Finally, we use~$\langle \it{ce^*}, \it{vn^*}, v_1, v_2 \rangle \MultipleSubstitutionRelation{} \langle v_1', v_2' \rangle$ to denote the update of evaluation functions~$v_1$ and~$v_2$ to~$v_1'$ and~$v_2'$, such that variables~$\it{vn^*}$ are mapped to the values~$\it{ce^*}$.
The first rule in Figure~\ref{fig:reusable-correct-transformations:transitions} deals with statements that send signals.
The evaluation functions~\vvars and~$\vvars'$ map values to variables and are used to evaluate the expressions that form the arguments of the signal.
The second rule in Figure~\ref{fig:reusable-correct-transformations:transitions} deals with conditional signal reception.
The rule specifies that the expression that forms the condition is evaluated using the possibly updated evaluation functions.
If the condition evaluates to {\bf true}, the transition is enabled.

%%%%%%%%%%%%%%%%%%%%%%%%%%
%%    State Machines
%%%%%%%%%%%%%%%%%%%%%%%%%%
\begin{figure}[hbt]
\centering
\fontsize{7pt}{0pt}{
$
\boxed{
\sosrule{
\it{trans} \in \it{trans^*}, \\
\langle \it{trans}, \ssms(\it{smn}), \vvars, \vsms(\it{smn}) \rangle
\TransitionsRelation{l}
\langle \it{sn}, \vvars', \vvars'' \rangle, \\
\ssms' = \ssms[\it{sn}/\it{smn}], \quad
\vsms' = \vsms[\vvars''/\it{smn}]
}{
\langle \it{smn}~\it{var^*}~\it{states}~\it{trans^*}, \ssms, \vvars, \vsms \rangle
\StateMachineRelation{l}
\langle \ssms', \vvars', \vsms' \rangle
}
}
$
}
\caption{Deduction rule for state machines}
\label{fig:reusable-correct-transformations:statemachine}
\end{figure}

The behavior of state machines is defined in terms of the behavior of their transitions, as shown in Figure~\ref{fig:reusable-correct-transformations:statemachine}.
The rule defines that if one of the transitions of state machine~$\it{sm}$ can perform an action represented by~$l$, then the state machine can perform the same~action.

%%%%%%%%%%%%%%%%%%%%%%%%%%
%%         Class
%%%%%%%%%%%%%%%%%%%%%%%%%%
\begin{figure}[hbt]
\centering
\fontsize{7pt}{0pt}{
$
\boxed{
\sosrule{
\it{sm} \in \it{sm^*}, \quad
\langle \it{sm}, \ssms, \vvars, \vsms \rangle
\xrightarrow{l}_{\it{SM}}
\langle \ssms', \vvars', \vsms' \rangle
}{
\langle \it{cn}~\it{var^*}~\it{port^*}~\it{sm^*}, \ssms, \vvars, \vsms \rangle
\xrightarrow{l}_{\it{CLASS}}
\langle \ssms', \vvars', \vsms' \rangle
}
}
$
}
\caption{Deduction rule for classes}
\label{fig:reusable-correct-transformations:classes}
\end{figure}

The rule in Figure~\ref{fig:reusable-correct-transformations:classes} shows how the enabled transitions, derived from the SOS rules for state machines, are lifted up to the level of a class.
It also shows implicitly that all state machines of a considered class are inspected.
The symbolic function~\ssms keeps track of the current states of the state machines, while~\vsms maps (local) variables to their value for the state machines of each object, and~\vvars maps the (global) variables at the level of the objects to their values.

%%%%%%%%%%%%%%%%%%%%%%%%%%
%%    Lists of Objects
%%%%%%%%%%%%%%%%%%%%%%%%%%
\begin{figure}[hbt]
\centering
\fontsize{7pt}{0pt}{
$
\boxed{
\begin{array}{c}
\sosrule{
\it{on_1}:\it{cn_1} \in \it{obj^*}, \quad
\it{cn_1}~\it{var^*_1}~\it{pn^*_1}~\it{sm^*_1} \in \it{class^*}, \\
\it{chn}(\it{type^*})~\textbf{async lossless}~\textbf{from}~\it{on_1}.\it{pn_1}~\textbf{to}~\it{on_2}.\it{pn_2} \in \it{chan^*}, \\
\langle \it{cn_1}~\it{var^*_1}~\it{pn^*_1}~\it{sm^*_1}, \sobjs(\it{on_1}), \vglob(\it{on_1}), \vloc(\it{on_1}) \rangle
\ClassRelation{\textbf{send}~\it{sgn}(\it{ce^*})~\textbf{to}~\it{pn_1}}
\langle \ssms, \vvars, \vsms \rangle, \\
\sobjs' = \sobjs[\ssms/\it{on_1}], \quad
\buf(\langle \it{chn}, \it{on_1}, \it{on_2} \rangle) = \textbf{nil}, \quad
\buf' = \buf[\langle \it{sgn}, \it{ce^*} \rangle/\langle \it{chn}, \it{on_1}, \it{on_2} \rangle]
}{
\langle \it{obj^*}, \it{class^*}, \it{chan^*}, \sobjs, \vglob, \vloc, \buf \rangle
\ObjectsRelation{\textbf{send}~\it{sgn}(\it{ce^*})}
\langle \sobjs', \vglob, \vloc, \buf' \rangle
}\\
\\
\sosrule{
\it{on_1}:\it{cn_1} \in \it{obj^*}, \quad
\it{on_2}:\it{cn_2} \in \it{obj^*}, \\
\it{cn_1}~\it{var^*_1}~\it{pn^*_1}~\it{sm^*_1} \in \it{class^*}, \quad
\it{cn_2}~\it{var^*_2}~\it{pn^*_2}~\it{sm^*_2} \in \it{class^*}, \\
\it{chn}(\it{type^*})~\textbf{sync}~\textbf{from}~\it{on_1}.\it{pn_1}~\textbf{to}~\it{on_2}.\it{pn_2} \in \it{chan^*}, \\
\langle \it{cn_1}~\it{var^*_1}~\it{pn^*_1}~\it{sm^*_1}, \sobjs(\it{on_1}), \vglob(\it{on_1}), \vloc(\it{on_1}) \rangle
\ClassRelation{\textbf{send}~\it{sgn}(\it{ce^*})~\textbf{to}~\it{pn_1}}
\langle \ssms, \vvars, \vsms \rangle, \\
\langle \it{cn_2} \, \it{var^*_2} \, \it{pn^*_2} \, \it{sm^*_2}, \sobjs(\it{on_2}), \vglob(\it{on_2}), \vloc(\it{on_2}) \rangle
\ClassRelation{\textbf{receive} \, \it{sgn}(\it{ce^*}) \, \textbf{from} \, \it{pn_2}}
\langle \ssms', \vvars', \vsms' \rangle, \\
\sobjs' = \sobjs[\ssms/\it{on_1}][\ssms'/\it{on_2}], \quad
\vglob' = \vglob[\vvars'/\it{on_2}], \quad
\vloc' = \vloc[\vsms'/\it{on_2}]
}{
\langle \it{obj^*}, \it{class^*}, \it{chan^*}, \sobjs, \vglob, \vloc, \buf \rangle
\ObjectsRelation{\it{sn}(\it{ce^*})}
\langle \sobjs', \vglob', \vloc', \buf \rangle
}
\end{array}
}
$
}
\caption{A subset of all deduction rules for compositions of objects}
\label{fig:reusable-correct-transformations:objects}
\end{figure}

Objects behave as specified by their class.
In a composition, objects participate and interact as described by the SOS rules for compositions of objects, some of which are given in Figure~\ref{fig:reusable-correct-transformations:objects}.
Every non-synchronizing transition of one of the objects enabled for execution in the current configuration is executed by the composition of the objects, and the functions are updated accordingly.
A non-synchronizing transition that receives signals over an asynchronous lossless channel is captured by the first rule in Figure~\ref{fig:reusable-correct-transformations:objects}.
The second rule describes synchronization of two objects via a synchronous channel.

%%%%%%%%%%%%%%%%%%%%%%%%%%
%%         Models
%%%%%%%%%%%%%%%%%%%%%%%%%%
\begin{figure}[hbt]
\centering
\fontsize{7pt}{0pt}{
$
\boxed{
\sosrule{
\it{m} \equiv \it{mn}~\it{obj^*}~\it{class^*}~\it{chan^*}, \\
\langle \it{obj^*}, \it{class^*}, \it{chan^*}, \sobjs, \vglob, \vloc, \buf \rangle
\ObjectsRelation{l}
\langle \sobjs', \vglob', \vloc', \buf' \rangle
}{
\langle \it{m}, \sobjs, \vglob, \vloc, \buf \rangle
\ModelRelation{l}
\langle \it{m}, \sobjs', \vglob', \vloc', \buf' \rangle
}
}
$
}
\caption{Deduction rule for models}
\label{fig:reusable-correct-transformations:models}
\end{figure}

Finally, the rule in Figure~\ref{fig:reusable-correct-transformations:models} defines the behavior of a model in terms of its objects, classes, and channels.
Each transition from one configuration to another is derived from the rules discussed above. 