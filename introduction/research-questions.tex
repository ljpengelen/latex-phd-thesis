\section{Research Questions}
\label{sec:introduction:research-questions}

We formulated a number of research questions aimed at resolving the problems described in Section~\ref{sec:introduction:problem-statement}.
The central research question is as follows.

%%%%%%%%%%%%%%%%%%%%%%%%%
%% Main research question
%%%%%%%%%%%%%%%%%%%%%%%%%

\RQMain

\noindent
This central research question is split into a number of more specific research questions.
Each of these questions is addressed in the remainder of this thesis.

%%%%%%%%%%%%%%%%%%%%%%%%%%%%%%%%%%%%%%%%%%%%%%%%%%%%%%%
%% Integrating textual and graphical modeling languages
%%%%%%%%%%%%%%%%%%%%%%%%%%%%%%%%%%%%%%%%%%%%%%%%%%%%%%%

To generate software from high-level descriptions, we first need to be able to produce models that form these descriptions.
During the Ideals project, we noticed that it was cumbersome to create large UML models using only graphical editors.
In search of a practical solution to this problem, we formulated the following research question.

\RQOne

%%%%%%%%%%%%%%%%%%%%%%%%%%%%%%%%%%%%%%%%%%%%%%%%%%%%%%%%%%%
%% Exploring the boundaries... Fine-Grained Transformations
%%%%%%%%%%%%%%%%%%%%%%%%%%%%%%%%%%%%%%%%%%%%%%%%%%%%%%%%%%%

In this thesis, we describe how software is generated from models on a high level of abstraction by first refining these models and then generating an implementation from the resulting model.
The original model is refined by applying a sequence of model transformations, where the application of each transformation leads to an intermediate model.
To be able to apply techniques for verification to as many of these intermediate models as possible, we formulated the following research question.

\RQTwo

%%%%%%%%%%%%%%%%%%%%%%%%
%% Prototyping semantics
%%%%%%%%%%%%%%%%%%%%%%%%

We developed a domain-specific modeling language, called the Simple Language of Communicating Objects~(\SLCO), and implemented a number of transformations that refine \SLCO models.
To be able to prove that these transformations preserve certain desirable properties of the source model, we first need to define the formal semantics of \SLCO.
Before giving a formal definition of the semantics of \SLCO, we wanted to experiment with a number of variations of the semantics.
To do so, we implemented an executable prototype of the semantics of \SLCO.
The following research question is related to this prototype.

\RQThree

%%%%%%%%%%%%%%%%%%%%%%%%%%%%%%%%%%%%%%%
%% Reusable and correct transformations
%%%%%%%%%%%%%%%%%%%%%%%%%%%%%%%%%%%%%%%

Generating reliable software by refining models is only possible if the model transformations used for the refinement preserve certain desirable properties of these models.
After formally defining the semantics of \SLCO, based on the aforementioned prototype, we investigated proving the correctness of a number of model transformations.
This led to the following research question.

\RQFour

%%%%%%%%%%%%%%%%%%%%%%
%% Evolution of a DSML
%%%%%%%%%%%%%%%%%%%%%%

Although we aimed at designing a DSML that is platform independent from the start, our DSML evolved over time.
To learn from our experiences in developing \SLCO and to be able to apply this knowledge while designing other DSMLs, we posed the following research question.

\RQFive

%%%%%%%%%%%%%%%%%%%%%%
%% LTS transformations
%%%%%%%%%%%%%%%%%%%%%%

Research question~\RQ{4} deals with a fixed set of model transformations.
We investigated how to prove the correctness of these transformations by means of proofs performed manually.
In contrast, the following research question is aimed at automated verification of model transformations.

\RQSix 