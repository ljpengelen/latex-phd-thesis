\section{Semantic Gaps and Platform Gaps}
\label{sec:slco:language-gaps}

There are a number of differences between \SLCO, \POOSL, \NQC, and \Promela in terms of the language constructs they offer.
These differences are often referred to as semantic gaps~\cite{Amstel2008}.
Furthermore, the Lego Mindstorms execution platform has practical limitations that poses restrictions to implementations in \NQC.
These restrictions form platform gaps because they do not hold for models specified using the other languages.
We identified a number of gaps between \SLCO and the target languages and platforms, which are shown in Table~\ref{tab:LanguageGaps}.
To successfully transform an \SLCO model into an equivalent model or implementation in one of the other languages, these gaps have to be bridged.

\begin{table*}[hbt]
\centering
\small
\begin{tabular}{|l|c|c|c|c|}
\hhline{~|----}
\rowcolor[gray]{.9}
\multicolumn{1}{l|}{}    & \textbf{\NQC}        & \textbf{\POOSL}     & \textbf{\Promela} & \textbf{\SLCO} \\
\hline
\bgc{(A)synchronous}     & \mr{2}{asynchronous} & \mr{2}{synchronous} & \mr{2}{both}      & \mr{2}{both} \\
\bgc{communication}      &                      &                     &                   & \\
\hline
\bgc{Reliability of}     & \mr{2}{unreliable}   & \mr{2}{reliable}   & \mr{2}{reliable}   & \mr{2}{both} \\
\bgc{communication}      &                      &                    &                    & \\
\hline
\bgc{Support for string} & \mr{2}{no}           & \mr{2}{yes}        & symbolic names for & \mr{2}{yes} \\
\bgc{constants}          &                      &                    & integer constants  & \\
\hline
\bgc{Connectivity for}   &                      &                    &                    & \\
\bgc{communication}      & \mr{-2}{broadcast}   & \mr{-2}{point-to-point} & \mr{-2}{point-to-point} & \mr{-2}{point-to-point} \\
\hline
\bgc{Number of objects}  & limited              & $\infty$           & $\infty$           & $\infty$ \\
\hline
\end{tabular}
\caption{Language and platform characteristics}
\label{tab:LanguageGaps}
\end{table*}

Each column lists the characteristics of one of the four languages and the corresponding platform.
The first row indicates whether communication is synchronous or asynchronous.
In case communication is synchronous, both the sender and receiver of a signal need to be available before a signal can be sent.
In this way, sender and receiver synchronize on communication.
In case communication is asynchronous, a sender can send a signal and proceed with its execution even though the receiver is not yet ready to receive the signal.
The second row indicates whether communication over channels is reliable.
In case a channel is reliable, which is also referred to as lossless, a signal that is sent will always arrive at the receiving end.
In case a channel is unreliable, which is also referred to as lossy, a signal that is sent may get lost.
The third row indicates whether a language supports string constants.
%Because \NQC does not support strings, each string constant present in an \SLCO model must be replaced by a unique integer constant.
Although \Promela has no support for strings, it offers an enumerated type that allows representing numeric constants using symbolic names, which can be regarded as a restricted form of string constants.
The fourth row shows whether signals are broadcasted or sent using point-to-point communication.
When signals are broadcasted, each signal can be received by multiple objects.
In the case of point-to-point communication, however, signals are sent from one object to exactly one other object.
The fifth row lists the amount of objects that can be instantiated simultaneously.
In \POOSL, \Promela, and \SLCO, this amount is unlimited.
For Lego Mindstorms, however, this number is limited in practice.
Because every object should be deployed on an RCX, the amount of concurrent objects is bounded by the available number of RCX controllers.
