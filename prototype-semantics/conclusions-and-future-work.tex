\section{Conclusions and Future Work}
\label{sec:prototype-semantics:Conclusions_and_Future_Work}

In this chapter, we addressed research question~\RQ{3} by implementing an executable prototype of the semantics of \SLCO using \ASFSDF.
We defined the semantics of \SLCO by implementing a number of tools that transform \SLCO models to representations of labeled transition systems, which has a number of advantages.
First, various existing tools for visualization of state spaces and verification can be reused because labeled transition systems are commonly used as input by such tools.
This provides the opportunity to apply these tools to verify and visualize \SLCO models, which aids the development of \SLCO itself as well as the related model transformations.
Furthermore, based on the work described in this chapter, we defined a formal semantics of \SLCO, which is described in Appendix~\ref{ap:sos-slco}.
The prototype of the semantics of \SLCO and its implementation as presented here provided a solid foundation for this work, as we got better understanding of the semantics of our DSML and were able to investigate a number of design decisions.

We implemented the transformation tools presented in this chapter using the \ASFSDFME.
In this way, we investigated its suitability for the purpose of implementing an executable prototype of the semantics of a DSML.
The biggest advantages of using the \ASFSDFME for this implementation are \ATerms~\cite{Brand:2007:AME:1219180.1219600}, used for representing terms, and the command-line tools that can be automatically generated.
The use of \ATerms guarantees efficient use of memory, and the command-line tools offer efficient execution of rewrite rules, without any additional effort during the implementation.
Both execution speed and efficient use of memory are important in this case because the state spaces of models represented by labeled transition systems are typically very large.
Unfortunately, active development of \ASFSDF has stopped.

As mentioned above, a formal semantics of \SLCO has been developed, which is based on the prototype described in this chapter.
However, the notion of time is not yet incorporated in the formal semantics of \SLCO as presented in Appendix~\ref{ap:sos-slco}.
Extending the formal semantics by including time is left as future research.
Additionally, we consider to apply the approach taken in this chapter to other DSMLs.
The approach lends itself well for the creation of state-space generators based on the operational semantics of a given DSML and prototyping the semantics of languages with informal or incompletely defined operational semantics. 