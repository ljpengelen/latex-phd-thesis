\section{Experimental Results}
\label{sec:case_studies:exp}

Table~XXX shows experimental results for five case studies with various rule systems\footnote{The rule system definitions can be found in Appendix~\ref{sec:appendix}.}, some preserving a relevant property (noted by \accept) and some not (noted by \reject).
The number of explored states and the runtime for full exploration are given for the {\it initial} model and the {\it trans}formed model.
The applied rule systems have been analysed separately ($\Sf$-{\it pres}.), and for these checks, the maximum number of states of the two LTSs involved in a check is given in the form ``(size left pattern)$+$(size right pattern)''.
Furthermore, the number of required checks, and the total runtime are reported.
The experiments have been performed on a machine with two dual-core \textsc{amd opteron} (tm) processors 885 2.6 GHz, 126 GB RAM, running \textsc{Red Hat} 4.3.2-7.
For divergence-sensitive branching bisimilarity checking, we used the {\it ltsconvert} tool of the \mCRLTwo toolset~\cite{mcrl2}.
The first three models are part of the distribution of \mCRLTwo.
We generated their LTSs with the \mCRLTwo tools, and manually transformed them to incorporate refined information concerning internal steps.
In the other two cases, synchronising behaviour was transformed, and the network LTSs have been constructed from sets of process LTSs using \EXPOPEN~\cite{lang05};
{\it brdcst} is a system of fifteen processes communicating via broadcast, i.e.\ three processes at a time synchronise simultaneously.
A practical transformation is to break this down into a series of two-party synchronisations, e.g.\ due to restrictions imposed for the eventual implementation.
We defined two rule systems for this, and they could be applied fifty times using a prototype tool developed by us.
The first of these rule systems does not preserve properties, whereas it can be shown that the second one does, using divergency information of the input models.
The {\it c.syst.} case is a communication system, where in five different places, communication between two processes is refined to use the ABP protocol, representing an adaptation to the use of lossy channels.
The different rule networks for the various checks were produced by our prototype, and we again used divergency information of the input models.
We analysed two versions of the rule system, one containing a subtle error (the receiver of messages does not expect messages with the wrong bit).
In this case, we expect that it is possible to define a $\Sf$-preserving rule system formalising the desired transformation, i.e.\ replacing one way of communicating by another should not affect the truth-value of properties not stating anything about the used communication mechanism.
Therefore, the negative outcome of a check is a strong indication that transformation of the network LTS results in a violation of the property.

The gain in speed is obvious, as is the gain in memory use, since it is linear to the number of analysed states.
In the future, we wish to look at larger, practical case studies to further validate the applicability of our techniques. 