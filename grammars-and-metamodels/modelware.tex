\section{Modelware}
\label{sec:grammars-and-metamodels:Modelware}

This section describes the implementation of our surface language using tools for model-to-text, text-to-model, and model-to-model transformations.
In this context, the term ``model'' refers to an instance of an explicit metamodel.
Tools that can perform transformations related to models are often referred to as \emph{modelware}.
Section~\ref{sub:grammars-and-metamodels:MW-Approach} describes our approach.
Section~\ref{sub:grammars-and-metamodels:MW-Implementation} describes the most important aspects of the implementation.
The tools used for the implementation are described in Appendix~\ref{ap:tools}.


\subsection{Approach}
\label{sub:grammars-and-metamodels:MW-Approach}

The rightmost part of Figure~\ref{fig:grammars-and-metamodels:approaches} gives a schematic overview of the approach using model-to-text (M2T), text-to-model (T2M), and model-to-model (M2M) transformations within a UML modeling tool.
The process of using modelware to transform a UML model containing fragments of surface language to a plain UML model can be divided into the following steps:
\begin{enumerate}
\item The fragments of surface language are extracted from the original model.
\item The extracted fragments are parsed and converted to a format usable by tools for model-to-model transformations.
\item The extracted and converted fragments of surface language are translated to equivalent Activities, as described in Section \ref{sub:grammars-and-metamodels:SL-Transformation}.
\item The fragments of surface language in the original model are replaced by the Activities created in the previous step.
\end{enumerate}
An advantage of this approach is that all transformations can be performed from within one and the same modeling environment.
In contrast to the approach described in Section \ref{sub:grammars-and-metamodels:GW-Approach}, no models have to be imported or exported during the transformation process.


\subsection{Implementation}
\label{sub:grammars-and-metamodels:MW-Implementation}
We used three tools for model transformation from the openArchitectureWare platform to implement the transformation described in Section \ref{sub:grammars-and-metamodels:SL-Transformation} following the approach described in Section~\ref{sub:grammars-and-metamodels:MW-Approach}.
These tools are described in Appendix~\ref{ap:tools}, and the implementation is described below.

We use \Xpand to extract fragments of surface language from models by traversing these models.
For each instance of \OpaqueBehavior in a model, the string describing its behavior is stored in a text file, including the name of the \OpaqueBehavior and the name of the \Class it is contained in.
Listing~\ref{lst:grammars-and-metamodels:Extracted-SL} shows the fragment of surface language extracted from the \OpaqueBehavior of Listing~\ref{lst:grammars-and-metamodels:Embedding-SL}.

\lstset{
    language=sl,
    label=lst:grammars-and-metamodels:Extracted-SL,
    caption=Extracted fragment of surface language,
    numbers=none
}

\begin{listing}
  \begin{lstlisting}
    behavior b C {
      return self
    }
  \end{lstlisting}
\end{listing}

We use \Xtext to parse and convert the extracted fragments of surface language to a format that is readable by \Xtend.
Because \Xtext uses \ANTLR, the class of textual representations that can be parsed is restricted to those that can be described by an LL(k) grammar.
A disadvantage of using \Xtext is that we had to modify our grammar for this reason.

One of the advantages of using the tools offered by the \OAW platform is their portability.
The platform is a collection of plug-ins for \Eclipse, and both \Eclipse and these plug-ins are available on a number of different operating systems.

Listing~\ref{lst:grammars-and-metamodels:Xtend-ex} shows a part of the transformation implemented in \Xtend from behavior modeled using our surface language to \Activities.
The listing shows that a new \ReadVariableAction, an \OutputPin, and an \ObjectFlow are created by defining local variables using \emph{let expressions}.
These expressions are followed by a \emph{chain expression}, which denotes the sequential evaluation of the expressions connected by the ``\texttt{->}'' symbols. The last two of these expressions use the \ObjectFlow to connect the \OutputPin of the \ReadVariableAction to the \InputPin of another \Action.

\lstset{
    language=xtend,
    label=lst:grammars-and-metamodels:Xtend-ex,
    caption=\Xtend extension that adds a \ReadVariableAction to an \Activity,
    numbers=none
}

\begin{listing}
  \begin{lstlisting}
    Void addReadVariableAction(
       uml::Activity a, uml::Package p, surfacelanguage::Variable v,
       uml::InputPin ip
    ) :
       let act = new uml::ReadVariableAction :
       let op = new uml::OutputPin :
       let of = new uml::ObjectFlow :
       a.node.add(act)
    -> a.edge.add(of)
    -> act.setResult(op)
    -> act.setVariable(v.createVariable(p))
    -> of.setSource(op)
    -> of.setTarget(ip)
    ;
  \end{lstlisting}
\end{listing} 