\section{Simple Transformation}
\label{sec:transformations-slco:simple}
The transformation~\TSim takes a model and a channel name as input.
If the channel is used for synchronous communication, it is replaced by an asynchronous channel, and the classes of the objects that are connected by the channel are adapted.
It is defined as
%
\[
\begin{array}{l}
\TSim(\it{mn}~\it{class^*}~\it{obj^*}~\it{chan^*}, \it{chn}) = \\
\quad \it{mn} ~
T^S_{\it{CS}}(\it{class^*}, \it{cn_1}, \it{pn_1}, \it{cn_2}, \it{pn_2}) ~
\it{obj^*} \\
\quad \it{chan_1^*}~\it{chn}(\it{type^*})~\textbf{async lossless between}~\it{on_1}.~\it{pn_1}~\textbf{and}~\it{on_2}.~\it{pn_2}~\it{chan_2^*},
\end{array}
\]
if~$\it{chan^*} \equiv \it{chan_1^*}~\it{chn}(\it{type^*})~\textbf{sync}~\textbf{from}~\it{on_1}.~\it{pn_1}~\textbf{to}~\it{on_2}.~\it{pn_2}~\it{chan_2^*}$, $\it{obj^*} \equiv \it{obj_1^*}~\it{on_1}:\it{cn_1}~\it{obj_2^*}$,
and~$\it{obj^*} \equiv \it{obj_3^*}~\it{on_2}:\it{cn_2}~\it{obj_4^*}$,
and it is defined as
\[
\TSim(\it{mn}~\it{class^*}~\it{obj^*}~\it{chan^*}, \it{chn}) = \it{mn}~\it{class^*}~\it{obj^*}~\it{chan^*}
\]
otherwise.

The transformation~$T^S_{\it{CS}}$ defined below takes a sequence of classes, two class names, and two port names as input.
It returns a sequence of classes.
The classes with the names provided as input are modified, and all other classes remain intact.
There are two cases for this transformation.
In one case, the class with name $\it{cn_1}$ is first in the sequence of classes.
In the other case, the class with name $\it{cn_2}$ is first.

\[
\begin{array}{l}
T^S_{\it{CS}}(\it{class^*}, \it{cn_1}, \it{pn_1}, \it{cn_2}, \it{pn_2}) =\\
\quad \it{class_1^*}~T^S_{\it{C}}(\it{class_1}, \it{pn_1})~\it{class_2^*}~T^S_{\it{C}}(\it{class_2}, \it{pn_2})~\it{class_3^*},
\end{array}
\]
if~$\it{class^*} \equiv \it{class_1^*}~\it{class_1}~\it{class_2^*}~\it{class_2}~\it{class_3^*}$, $\it{class_1} \equiv \it{cn_1}~\it{pn_1^*}~\it{var_1^*}~\it{sm_1^*}$, and $\it{class_2} \equiv \it{cn_2}~\it{pn_2^*}~\it{var_2^*}~\it{sm_2^*}$, and
\[
\begin{array}{l}
T^S_{\it{CS}}(\it{class^*}, \it{cn_1}, \it{pn_1}, \it{cn_2}, \it{pn_2}) = \\
\quad \it{class_1^*}~T^S_{\it{C}}(\it{class_2}, \it{pn_2})~\it{class_2^*}~T^S_{\it{C}}(\it{class_1}, \it{pn_1})~\it{class_3^*},
\end{array}
\]
if~$\it{class^*} \equiv \it{class_1^*}~\it{class_2}~\it{class_2^*}~\it{class_1}~\it{class_3^*}$, $\it{class_1} \equiv \it{cn_1}~\it{pn_1^*}~\it{var_1^*}~\it{sm_1^*}$, and~$\it{class_2} \equiv \it{cn_2}~\it{pn_2^*}~\it{var_2^*}~\it{sm_2^*}$.

The transformation~$T^S_{\it{C}}$ defined below takes a class and a port name as input and returns a class with modified state machines.
Only the state machines that send or receive signals over the given port are modified.
%
\[
T^S_{\it{C}}(\it{cn}~\it{pn^*}~\it{var^*}~\it{sm^*}, \it{pn}) = \it{cn}~\it{pn^*}~\it{var^*}~T^S_{\it{SMS}}(\it{sm^*}, \it{pn})
\]

The transformation~$T^S_{\it{SMS}}$ defined below takes a sequence of state machines and a port name as input, and returns a modified sequence of state machines.
Only the state machines that send or receive signals over the given port are modified.
There are two cases for this transformation.
One case deals with empty sequences of state machines and one case with non-empty sequences of state machines.
%
\begin{align*}
T^S_{\it{SMS}}(\epsilon, \it{pn}) & = \epsilon \\
T^S_{\it{SMS}}(\it{sm}~\it{sm^*}, \it{pn}) & = T^S_{\it{SM}}(\it{sm}, \it{pn})~T^S_{\it{SMS}}(\it{sm^*}, \it{pn})
\end{align*}

The transformation~$T^S_{\it{SM}}$ defined below takes a state machine and a port name as input, and returns a state machine with modified transitions and a number of additional states.
Only the transitions that send or receive signals over the given port are modified.
There are two cases for this transformation.
One case deals with state machines without final states, and the other with state machines with final states.
%
\[
T^S_{\it{SM}}(\it{smn}~\it{var^*}~\textbf{initial}~\it{sn}~\it{sn^*}~\it{trans^*}, \it{pn}) =  \it{smn}~\it{var^*}~\textbf{initial}~\it{sn}~\it{sn^*}~\it{sn^*_1}~\it{trans^*_1},
\]
where~$\langle \it{sn^*_1}, \it{trans^*_1} \rangle = T^S_{\it{TS}}(\it{trans^*}, \it{pn})$, and
\[
\begin{array}{l}
T^S_{\it{SM}}(\it{smn}~\it{var^*}~\textbf{initial}~\it{sn}~\it{sn^*}~\textbf{final}~\it{sn^+}~\it{trans^*}, \it{pn}) = \\
\quad \it{smn}~\it{var^*}~\textbf{initial}~\it{sn}~\it{sn^*}~\it{sn^*_1}~\textbf{final}~\it{sn^+}~\it{trans^*_1},
\end{array}
\]
where~$\langle \it{sn^*_1}, \it{trans^*_1} \rangle = T^S_{\it{TS}}(\it{trans^*}, \it{pn})$.
The syntax definition given in Appendix~\ref{ap:sos-slco} is extended as follows to accommodate final states.
%
\[
\it{sm} ::= \it{smn}~\it{var^*}~``\textbf{initial}"~\it{sn}~\it{sn^*}~[``\textbf{final}"~\it{sn^+}]~\it{trans^*}
\]


The transformation~$T^S_{\it{TS}}$ defined below takes a sequence of transitions and a port name as input, and returns a modified sequence of transitions and a set of new states.
Only the transitions that send or receive signals over the given port are modified.
There are two cases for this transformation.
One case deals with empty sequences of transitions and the other with non-empty sequences.
%
\begin{align*}
T^S_{\it{TS}}(\epsilon, \it{pn}) & = \langle \epsilon, \epsilon \rangle \\
T^S_{\it{TS}}(\it{trans}~\it{trans^*}, \it{pn}) & = \langle \it{sn^*_1}~\it{sn^*_2}, \it{trans^*_1}~\it{trans^*_2} \rangle,
\end{align*}
where~$\langle \it{sn^*_1}, \it{trans^*_1} \rangle = T^S_{\it{T}}(\it{trans}, \it{pn})$ and~$\langle \it{sn^*_2}, \it{trans^*_2} \rangle = T^S_{\it{TS}}(\it{trans^*}, \it{pn})$.

The transformation~$T^S_{\it{T}}$ defined below takes a transition and a port name as input, and returns a set of states and a set of transitions.
Only transitions that have a signal reception trigger or a send signal statement are replaced by new transitions.
All other transitions are left unaltered.
There are three cases for this transformation.
One case deals with transitions that send a signal, one case deals with transitions that receive signals, and one case deals with all other transitions.
%
\[
\begin{array}{l}
T^S_{\it{T}}(\it{tn}~\textbf{from}~\it{sn_1}~\textbf{to}~\it{sn_2}~\textbf{send}~\it{sgn}()~\textbf{to}~\it{pn}, \it{pn}) =
\langle \\
\quad \it{sn_3}, \\
\quad \it{tn_1}~\textbf{from}~\it{sn_1}~\textbf{to}~\it{sn_3}~\textbf{send}~\it{sgn_1}()~\textbf{to}~\it{pn} \\
\quad \it{tn_2}~\textbf{from}~\it{sn_3}~\textbf{to}~\it{sn_2}~\textbf{receive}~\it{sgn_2}()~\textbf{from}~\it{pn} \\
\rangle,
\end{array}
\]
where~$\it{sn_3}$ is a fresh state name, $\it{tn_1}$ and~$\it{tn_2}$ are fresh transition names, $\it{sgn_1}~\equiv~``s\_"+\it{sgn}$, and $\it{sgn_2}~\equiv~``a\_"+\it{sgn}$,
\[
\begin{array}{l}
T^S_{\it{T}}(\it{tn}~\textbf{from}~\it{sn_1}~\textbf{to}~\it{sn_2}~\textbf{receive}~\it{sgn}()~\textbf{from}~\it{pn}, \it{pn}) =
\langle \\
\quad \it{sn_3}, \\
\quad \it{tn_1}~\textbf{from}~\it{sn_1}~\textbf{to}~\it{sn_3}~\textbf{receive}~\it{sgn_1}()~\textbf{from}~\it{pn} \\
\quad \it{tn_2}~\textbf{from}~\it{sn_3}~\textbf{to}~\it{sn_2}~\textbf{send}~\it{sgn_2}()~\textbf{to}~\it{pn} \\
\rangle,
\end{array}
\]
where~$\it{sn_3}$ is a fresh state name, $\it{tn_1}$ and~$\it{tn_2}$ are fresh transition names, $\it{sgn_1}~\equiv~``s\_"+\it{sgn}$, and~$\it{sgn_2}~\equiv~``a\_"+\it{sgn}$, and
\[
\begin{array}{l}
T^S_{\it{T}}(\it{trans}, \it{pn}) = \langle \epsilon, \it{trans} \rangle,
\end{array}
\]
for all other transitions. 