\section{Experimental Results}
\label{sec:lts-transformation:experiments}

\begin{table}[hbt]
\centering
\scriptsize
\begin{tabular}[r]{|cl|rrrrrrr|}
\hhline{~~|-------}
\mc{2}{l|}{}                                      & \head{ACS}         & \head{1394-fin} & \head{wafer} & \dhead{broadcast}      & \dhead{ABP} \\
\hline
\cellcolor[gray]{0.9}         & \cat{\#states}    & \mc{1}{r}{4,764}   & 188,569         & 78,919       & 161,051   & 161,051    & 759,375     & 759,375 \\
\mr{-2}{\bgc{initial model}}  & \catu{time}{sec.} & \mc{1}{r}{1.85}    & 379.08          & 4.88         & 3.48      & 3.48       & 29.97       & 29.97 \\
\hline
\bgc{transformed}             & \cat{\#states}    & \mc{1}{r}{21,936}  & 5,849,124       & 375,937      & 4,084,101 & 28,629,151 & 656,356,768 & 656,356,768 \\
\bgc{model}                   & \catu{time}{sec.} & \mc{1}{r}{10.23}   & 18,045.13       & 49.33        & 83.53     & 952.85     & 48,795.28   & 45,553.27 \\
\hline
\cellcolor[gray]{0.9}         & \cat{max.\ \#st.} & \mc{1}{r}{2+3}     & 2+6             & 2+5          & 27+30      & 27+31     & 15+58       & 15+58 \\
\bgc{preservation}            & \cat{\#checks}    & \mc{1}{r}{1}       & 1               & 1            & 7          & 7         & 63          & 63 \\
\bgc{checks}                  & \cat{result}      & \mc{1}{r}{\accept} & \accept         & \accept      & \reject    & \accept   & \reject     & \accept \\
\cellcolor[gray]{0.9}         & \catu{time}{sec.} & \mc{1}{r}{0.01}    & 0.01            & 0.01         & 0.616      & 0.792     & 1.90        & 1.90 \\
\hline
\end{tabular}
\caption{Experimental results}
\label{tab:lts-transformation:results}
\end{table}

Table~\ref{tab:lts-transformation:results} shows experimental results for five case studies with various rule systems.
%%%%%%%%%%, some preserving a relevant property (noted by \accept) and some not (noted by \reject).
The number of explored states and the runtime for full exploration are given for the initial model and the transformed model.
The applied rule systems have been analyzed using the proposed technique.
For the resulting checks, the maximum number of states of the two LTSs involved in a check is given in the form ``(size left pattern)$+$(size right pattern)''.
If a rule system is property preserving, this is denoted by \accept in the table.
The result of an unsuccessful preservation check is denoted by \reject.
Furthermore, the number of required checks and the total runtime are reported.
We report the time needed to perform a complete state-space exploration because this provides an indication of the amount of time required to (re)check any given property.
The experiments have been performed on a machine with two dual-core \textsc{amd opteron} (tm) processors 885 2.6 GHz, 126 GB RAM, running \textsc{Red Hat} 4.3.2-7.
For DSBB checking, we used the \ltscompare tool of the \mCRLTwo toolset~\cite{mcrl2}.
The results clearly show that checking property preservation takes much less time than exploration of the state space of transformed models.

The first three models are part of the distribution of \mCRLTwo.
We generated their LTSs with the \mCRLTwo tools and manually transformed them to incorporate refined information concerning internal steps.
The applied transformation is similar to the one shown in Figure~\ref{fig:lts-transformation:preserving-rules}.
In the other two cases, synchronizing behavior was transformed, and the network LTSs have been constructed from sets of process LTSs using \EXPOPEN~\cite{lang05}.
The case study named {\it broadcast} concerns a system of five sets of three processes communicating via broadcast.
The three processes in each set synchronize simultaneously.
The rule systems we applied break this down into a series of two-party synchronizations.
We defined two rule systems for this, and they could be applied fifty times using a proprietary prototype tool.
The first of these rule systems does not preserve properties, and the second one does.
The case study named {\it ABP} concerns a system in which communication between two processes is refined to use the alternating bit protocol~\cite{Bartlett1969} in five different places.
The rule networks for the various checks were produced by our prototype.
We analyzed two versions of the rule system, one containing the subtle error that the receiver of messages does not expect messages with the wrong bit.
A more detailed description of the case studies is available in Appendix~\ref{ap:case_studies}. 